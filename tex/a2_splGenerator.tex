To increase the number of subject systems and
inspect how each evaluation strategy behaves with the growth of the
configuration space, we implemented a product-line generator tool called
SPL--Generator\footnote{\url{https://github.com/SPLMC/spl-generator/}}, which is
able to create a software product line from scratch or modify an existing one by
incrementally adding features and behavior to its models. For the feature model
generation (i.e., to create a new feature model or change an existing one), the
tool relies on the SPLAR tool~\cite{mendonca:2009}. The desired characteristics
of the resulting feature model are obtained by defining accordingly the set of
parameters provided by SPLAR. Examples of such parameters are the number of
features to be created, the amount in percentage for each kind of feature
(mandatory, optional, OR-inclusive and OR-exclusive), and the number of
cross-tree constraints. As our SPL-Generator tool intends to create product
lines that resemble real-world product lines, it produces only consistent
feature-models (i.e., the SPLAR's parameter for creating consistent
feature-models is always set to \texttt{true}).

To create behavioral models, the SPL-Generator tool considers the UML behavioral
diagrams and follows the refinement of activity diagrams into sequence diagrams
presented in Section~\ref{sec:runningExample}. For creating activity  and
sequence diagrams, the generator tool is also guided by a set of parameters for
each kind of behavioral diagram.  For an activity diagram, it is possible to
define how many activities it will comprise, the number of decision nodes, and
how many sequence diagrams will refine each created activity. For a sequence
diagram, it is possible to define its size in terms of numbers of behavioral
fragments, the size of each behavioral fragment in terms of the number of
messages, the number of lifelines, the number of different reliability values
(such that each lifeline will  randomly assume only one value) and the range for
them. Thus, one possibly generated  sequence diagram  would have 5 behavioral
fragments, each one containing 8 messages between 3 lifelines, whose reliability
values are within the range $[0.99, 0.999]$. 

Finally, the SPL-Generator tool also provides a parameter to define how the
feature model and the behavioral models will be related.  The allocation of a
behavioral fragment (implementing a feature's behavior) can be fully
\emph{randomized} within the set of created sequence diagrams, or it can be
\emph{topological}, which means the relations between the behavioral fragments
mimic the relations between the corresponding features.  In the latter, we
assume a child feature refines its parent, so its behavioral fragment is nested
into its parent's behavioral fragment.

